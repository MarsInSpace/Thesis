The term "Flow" has become a very popular word, often used within the gaming industry. Flow is usually defined as a state of concentration so focused that it leads to absolute absorption in an activity \cite{higher}. Therefore, the "Flow Zone" is the zone where a person is engaged to the fullest \cite{mmo} and is therefore a very valuable and desirable state for game developers to have their games induce in their users.

Flow can come in many different forms, however a base set of behavioural patterns can be observed in individuals that enter a state of flow. These can include the ability to complete tasks, concentrate deeply, have clear goals, achive effortless involvement, to control one's actions, and have no concern for self or a sense of time during the activity. In short, it produces states of enjoyment represented by deep concentration on the activity \cite{higher}.

The state of flow, or flow zone is easily recognisable, but how is it induced? What are the conditions that need to be met for a person to enter such a state? A couple of strong motivators and triggers have been identified that could facilitate the entry.
Deep concentration can happen as a result of balanced but interesting challenges and use of skills, as well as a sense of control and satisfaction in the task \cite{engage} \cite{equilibrium}.
The activity needs to be sufficiently challenging and has a progression in difficulty to keep the user activated at all times \cite{higher} \cite{equilibrium}.
Clear goals at every step, immediate feedback, activities that merge action and awareness, a lack of distractions, no worry of failure or punishment due to failure are key factors that can help achieve Flow \cite{mmo} \cite{equilibrium}.
A mix of intrinsic and extrinsic motivators help the user stay engaged \cite{mmo}, a general intrinsic sense of reward can facilitate this even further \cite{equilibrium}.
In education, flow can be achieved through a combination of well-presented knowledge, interested students and stimulating teachers \cite{higher}.

In gamification, flow can be seen as a very important aspect of making the design work and effectively motivate the user.
The intrinsic characteristics of flow are argued to be present in gamification already, encouraging engaged and focused behaviour \cite{higher}.
In general, flow is an influential motivational theory often applied in gamification design and has frequently been used to create engaging, enjoyable experiences that captivate the users attention \cite{equilibrium}.
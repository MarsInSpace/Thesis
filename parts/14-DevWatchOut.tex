There are many pitfalls in the design and development process of a gamified application for personal learning purposes. These can mostly be compiled into the following categories: the differences between learning in the classroom and individual learning; game design versus instructional design; the target demographics of the application; and general learner motivation and needs.

Learning within a controlled classroom environment, be it at school, university, or similar, is very different from learning by yourself from home. Due to the 2020 pandemic, many people had to switch to online learning and several issues have been observed, mainly the lack of motivation or the inability to properly self-motivate, as well as a lack of structure \cite{online}. A teacher provides a schedule, a system to learn, support when needed, and the presentation of the learning materials. All this needs to be substituted skillfully so the user does not feel overwhelmed or lost and can profit from an effective learning process.

This leads to one of the main issues within the development of gamified instructional material. The design of these tools needs to balance pedagogical requirements and learning aspects with the fun-factor of games to function effectively, which is difficult as even in commercial game development triggering that notion of fun for the player while gaming can be tricky \cite{online}.
Often, gamified tools for learning are either designed by a team of instructional designers, or a team of game designers, but rarely both. Studies show that for successful development, both game designers and educators are needed to contribute good and engaging game design but also good pedagogy, otherwise the resulting product will neither be fun nor motivating, and additionally lacking in the application of key principles that facilitate effective learning \cite{framework}.

Demographic dependence is another big aspect to keep track of in the development process. It is important to know what target group you are catering to and what their individual needs are. Those demographic groups can include but are not limited to: gender, age, ethnicity, social background, disability, and generally minorities and disadvantaged communities.
Age and gender for example play a major role in which gamification mechanics specifically are efficient and well recieved \cite{fail}. For instance, competition and competitive play is differently recieved between people of different genders, regardless of age. Goals and appeal to emotions have different importance for different age groups, as well as the reception and motivating factor of different types of rewards.
Depending on the purpose of the application and its users intended social environment, gamification tools need to be adapted to those differing needs as well. If the tool is designed for the use in the workplace, specifically for larger companies, competitive elements that promote competition between co-workers might rather be percieved as a stress factor and even as threatening instead of motivating \cite{lifelong}. If the tool is designed for private use, some friendly competition with registered friends can go a long way in motivating the user instead of being a stressor.
It is important to research previously conducted studies about the preferences and psychology of the target demographic that the application is aimed at to get a good grasp on what approach to take in the style of gamification which should be used in development.
Like in any other medium, but especially in educational and learning contexts, the representation and portrayal of minorities and disadvantaged communities is a sensitive subject and a hard terrain to navigate. Discrimination because of race, gender, disability, or social status is increadibly common and sometimes done not out of malice but out of lack of knowledge about these topics. It is important to be mindful and respectful of the differences and needs of these groups, especially if no developer that is part of the affected groups is part of the team \cite{engage}. Again, research is of utmost importance to create an inclusive and welcoming learning environment, also within gamified learning tools.

Lastly, paying attention on how to properly captivate the users attention and foster motivation and engagement through gamification mechanics is key. Being overzealous in the use of gamification aspects can end up being more harmful than effective. The quality of motivation is much more important than the quantity of it \cite{equilibrium}. Therefore, understanding nuances between intrinsic and extrinsic motivation, how to trigger them, and the strengths and weaknesses of both those approaches, is crucial to use them effectively in the design, development, and implementation of the gamified tool. It is important to keep in mind the 'Overjustification Effect' as well - the paradox that when extrinsic rewards are provided for intrinsically motivated activities, intrinsic motivation decreases \cite{equilibrium}. Thorough reflection on which rewards and motivators are being used is necessary to create a successful gamified learning experience.

It is important to keep in mind that not all kinds of rewards and gamification mechanics work for all learners and for all learning outcomes. Being aware of the target group and of the ultimate goal for which the gamified application is being developed will provide the required insight to choose the right game elements to utilize \cite{framework}.
Solely focussing on flow theory may not guarantee the wanted outcomes in learning and habit building, it is needed to apply additional motivational frameworks to achieve the desired result \cite{equilibrium}. Instead, finding a good balance between challenge and skill encourages the neglect of distractions and helps foster concentration, engagement and enjoyment, which in turn can also trigger Flow \cite{higher}. The level of difficulty has to be variable enough to adjust to the individual users abilities and needs to avoid a negative impact on their motivation, allowing them to succeed and keep progressing all while maintaining a healthy degree of pleasantly frustrating challenge that triggers the users ambition \cite{model}. Walking the line of keeping the application sufficiently challenging while not making it too demanding or stressful can be difficult, although it is crucial to not make the user percieve the usage of it as work or as unenjoyable \cite{mmo}.

The so called 'BPL Gamification' is a gamification system often refered to as 'Pointification' by experts in the field and refers to the shallow gamification technique of using badges, points and leaderboards to create an externally appealing combination of game elements and is wrongly assumed to make the product immediately more engaging by applying it to a process \cite{equilibrium}. This process of arbitrarily rewarding users with points and similar items is massively overused and not very effective in most cases. In the worst case, such poorly applied game design elements can undermine a learners innate motivation to partake in an activity and sabotage the initial goal of using gamification in the tool that is being developed. Instead it is recommended to implement deep gamification techniques, focusing beyond the superficial focus on extrinsic rewards, to foster an immediate sense of gratification in the user and additionally offer effective long-term learning outcomes.

Not falling into the trap of desiring to benefit from the positive effects and outcomes of gamification without putting in the proper time, research and effort to develop a strategy fitting to your own project is the path to a successful gamified product, since according to many empirical findings, a simple BPL approach is not enough for creating an effective gamified experience \cite{equilibrium}.
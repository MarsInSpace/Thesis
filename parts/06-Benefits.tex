Video games have great potential in addition to their entertainment value.
There has been considerable success when games were designed to address a specific problem or to teach their players a certain skill \cite{compare}, which sparked interest to use more of this medium in regular instances of education.
Digital media and games in a teaching and learning context have been considered a valuable tool to engage students and make the learning process more interesting and motivating. Empirical research has shown that such games succeed in their goal to provide learning benefits \cite{framework} \cite{compare} \cite{domestic}.
games have always fascinated people in general \cite{aspects}, and due to the current generation of learners - mostly Millenials, Generation Z and Generation Alpha - being digital natives, which means they grew up with modern technology and are therefore already accustomed to digital games and content \cite{edu} \cite{domestic}, the further increase of such media being present in every context including the educational sector is the next logical step that can already be observed in day-to-day life.

Through simple educational tasks, studies have demonstrated that learning within a motivating setting improves learning outcomes and facilitate learning in general through enhanced engagement \cite{aspects} \cite{edu}.
Learning through gamification or other playful, mostly digital techniques gives a break from other more traditional teaching methods and embraces the motivational properties of such newer teaching frameworks. Studies show that digital games and media in general seem more attractive and enjoyable to students than traditional instruction through books or simply listening to a teacher speak \cite{compare} \cite{domestic}.

Old teaching methods have mechanisms that are no longer beneficial to students in the current day due to several reasons \cite{compare}.
Such reasons include but are not limited to:
\begin{itemize}
\item The methods do not encourage students to think outside the box or to find their own ways of coming to a conclusion since usually a very specific and rigid path to come to a solution is being taught \cite{compare}.
\item In school and other academic educational environments, the focus of the studies tends to be on the aquisition of good grades and on the passing of exams instead of encouraging the student to understand underlying concepts of the subject, which can lead to the student studying isolated topics very intensely just before an exam period and afterwards discarding the knowledge instead of integrating it effectively within their existing knowledge base \cite{compare}.
\item It tends to ignore the importance of integrating current and new trends of our society and culture into the educational environment, which causes a lack in authenticity in the assigned tasks and misses the realism that is necessary to allow students to be able and properly use the aquired skills outside of an educational context \cite{compare}.
\item Individual coaching and the organic adaptation of the learning material to cater to specific student needs is difficult to execute and non-existent in most educational contexts, especially for people with special needs. Although it is key in having a successful learning process, the rigidity of the current teaching methods make little room for this \cite{traditional}.
\end{itemize}

Generally, introducing gamification and Game-Based Learning in learning contexts renders the entire experience more fun which makes it more effective for most users \cite{aspects}.
Studies have shown that games in educational settings provide immersion, motivation and a high level of engagement as well \cite{framework}.
Besides just the fun-factor, several aspects of the learning process are being supported and reinforced which leads to increased effectivity.
Such aspects include:
\begin{itemize}    
\item The encouragement for intersectionality in the learning topic and/or process. Users are encouraged to combine knowledge from different areas of knowledge and expertise to choose a solution, make a decision or pick a way forward \cite{aspects}.
\item The freedom to test different outcomes or solutions to a problem based on the users decisions and actions, immediate feedback providing satisfying cues to continue trying and encouraging the exploration of different perspectives on the same problem \cite{aspects}.
\item In some instances, users are encouraged to contact other users or team members to discuss and negotiate what steps to take, which teaches and reinforces improvisation in tricky situations as well as social skills \cite{aspects}.
\item It promotes the users problem-solving abilities. Without the fear of severe failure, the user has the freedom to try out many different ways of solving problems, helping to create a more sound and solid approach to problem-solving tasks in the future \cite{compare}.
\item It equally promotes reflection and the exploration of the topic and of the user themselves, allowing them to progress at their own pace and take as much time as they need to repeat the different tasks and topics, creating a deeper space to integrate the learned themes \cite{engage}.
\item A platform for the users creative thought is provided and it creates a safe space to foster innovation without fear of failure \cite{compare} \cite{traditional}.
\end{itemize}

Gamification can create a mindset that encourages the user to try new things without suffering from a fear of failure and generally creates an enjoyable mental space and environment that promotes learning \cite{engage} \cite{compare}.
According to Gabe Zichermann, the use of game mechanics in the teaching of a new skill improves the ability of aquiring said skill by as much as 40\% \cite{edu}.
    
Additionally to reinforcing important skills such as the ones named above and the general learning of the subject matter, gamified and game-based learning teaches many other important skills and properties like digital fluency, automisation through repetition and general 21st century skills that are important for navigating the modern world \cite{framework} \cite{domestic}.
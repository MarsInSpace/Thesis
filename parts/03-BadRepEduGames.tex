Although the newer generations are avid tech users who navigate the digital world on a daily basis, educational games are associated with negative feelings and are often seen as boring in the best case \cite{model} \cite{domestic}, or outright useless or even damaging in the worst \cite{domestic} \cite{lifelong}.
Critical voices claim that learning through games simply disguises traditional teaching and therefore won't be enjoyable in and of itself in the first place \cite{domestic}.
Additionally, using games within learning contexts usually have somewhat of an unserious flavour to it, especially from the perspective of formal educational systems since games are still percieved as a mere entertainment medium \cite{lifelong}.
Another reason for the bad reputation of educational games might be the ongoing debate whether games in general are harmful to players, for example due to some titles' extreme violence like Call of Duty, published by Activision or the Counter Strike franchise, by Valve Corporation, rendering games unattractive as a medium for learning in school contexts \cite{domestic}.
The question of why no actually good and fun educational games have been developed yet is a widely spread one, and can be traced back to the roots of development. Both game designers and educators or educational designers are required in the development process to make a functioning and successful learning experience. If only Educators are used to make such a tool, the educational game is usually designed from their content as the base to work off of, with the 'actual' game aspects being added further down the line which leads to ineffective game design, loss of motivation and engagement \cite{framework} and therefore makes the learning experience unenjoyable due to the lack of knowledge of how to make a game properly fun \cite{online}.
Vice versa, if only game designers are part of the process and don't heed the experience and expertise of people actually working in the educational environment, key pedagological principles vital for effective learning might be neglected to be implemented in the game, rendering the game fun, but ineffective as a learning tool \cite{online}.
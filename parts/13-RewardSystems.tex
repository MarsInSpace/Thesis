Reward systems are the most important design element for any game and especially any gamified application. They keep the users engaged and motivated, appealing to their sense of ambition and need for affirmation. Previously, the concepts of intrinsic and extrinsic motivation were defined and analysed, most reward systems build on top of that knowledge and aim to trigger said kinds of motivation.

One reward system that has been proposed is called 'The Three corners of Reward' \cite{corners}. It divides the different types of rewards into three archetype categories, explores how to combine these for efficient use, and analyses their interactions with each other. The main three categories are personal reward, material reward and competitive reward.

\textbf{Personal Reward:}
The players already go into the game with their own ideals and goals that are the main reason they are playing the game for. These goals are personal to the player which makes them volatile and therefore hard to study and define well, however certain core values can be extracted and observed. These core values usually are for example ambitions to unlock something inside the game, or the completion of the game.

\textbf{Material Reward:}
A material reward is a reward given for the players actions, and for their successful behaviour in game. The player obtains something they want after putting time and effort into playing the game, this often is an item, experience points or temporary improvement that gives them an advantage.

\textbf{Competitive Reward:}
The competitive reward is often described through the satisfaction after besting another player. This can mean getting a higher score, a quicker time, beating other players in battle or just having a higher score of completion in the game.

Most often, games offer a combination of these types of rewards to trigger different types of motivations within the player, namely intrinsic motivation, extrinsic motivation and social play as a motivator.

\textbf{Personal/Material:} 
The interaction of these reward systems result in triggering intrinsic motivation within the player. It promotes genuine enjoyment in the activity and creating a large interest to play in the player with little to no external influence. The enjoyment comes from personal aspects and the game becomes more meaningful to the player. This increases the better the player gets at the game. This type of reward is generally favoured due to its fast pace and is often used in recent popular game genres to create enjoyment on a short term basis.

\textbf{Material/Competitive:}
The interaction of the material reward and competitive reward results in enhanced social play motivation. It offers ways for players to connect and play with others in a multiplayer environment. This increases the general competitiveness of the players as well as the overall time spent playing the game.

\textbf{Personal/Competitive:}
Personal and competitive rewards combine to create extrinsic drivers to keep the player engaged. It is a motivation to win in order to receive an outcome. This could for example be scoring well on a public leaderboard, since this would be viewed as an attractive position to dedicated players. However, extrinsic rewards are usually very time consuming to achieve and tend to be unfocused in which kinds of players they appeal to, making it important to not rely on this area alone to keep the players engaged. The rewarding feeling tends to not last as long as the other types of rewards, which can further be unappealing to players. It also happens that players create high expectations for themselves, resulting in more negative thoughts and feelings towards potential failure than usual.

Another system suggests to not just differentiate intrinsic and extrinsic motivation but offers more concrete aspects within extrinsic motivation, separating them further into underlying types. These include the following \cite{equilibrium}:

\textbf{External Regulation:}
External regulation is defined as the least autonomous form of extrinsic motivation. The main driver and motivator here is the obtaining of a reward, or the avoidance of a penalty. Players engage in the demanded behaviours to maximise the pleasure of recieving an external reward and to minimise the chance of recieving punishment for a failed task. Points, badges and leaderboards can be used to motivate the user in that regard, promoting engagement in the task at hand.

\textbf{Introjected Regulation:}
Internal pressure like guilt or shame is a very powerful motivator for people. Introjected regulation relates to this feeling of pressure to motivate the user. The actions and tasks that need to be completed are still imposed externally, however the source of pressure and therefore motivation comes from the user themselves. Gamified systems are capable of harnessing this by using simple mechanisms like the common and well known 'Streak'-system as presented in Figure \ref{fig:2}, generating the internal pressure within the user to keep logging in each day to maintain this streak and can act as a powerful motivator to promote continued use of the application.

\textbf{Identified Regulation:}
A more autonomous form of extrinsic motivation is identified regulation. In this type of motivation, the user individually recognizes the value of the promoted behaviour and accepts that motivation as their own. This happens when the users personal goals are aligned with the goals of the application, for example learning a new skill that they need to learn for personal purposes. Gamification can shape the given experience within the gamified tool to match these personal goals, although that requires a good grasp and knowledge about the user base.

\textbf{Integrated Regulation:}
Integrated regulation is the most autonomous form of extrinsic motivation. When the individual users general life goals and their identity aligns with the goals of and desired behaviour asked by the gamified application, a very deep and powerful motivation can be formed. If motivation like that can be achieved, it fosters deeper engagement and mastery of the subject at hand. By promoting general skills that align with the users overarching goal and personal development, for example in creativity or problem-solving, gamification can help reach that level of profound engagement.

Besides just systems that aim to narrow down the types of rewards and how to properly make use of them in game development and the development of gamified applications, there also are more specific rewards that are commonly given within games and have proven themselves to be successful \cite{mmo}.
Experience point Rewards are a universal way of keeping players motivated and ignite their ambition to keep playing and improving in the game. This is usually portrayed through an experience point meter that shows how many more the player needs to accumulate to advance to the next level. This goes hand in hand with another popular reward, namely Unlocking Mechanisms. Through for example gathering experience and advancing within the game or application, new parts of the game can be accessed, like new maps, new levels or new item rewards, that were previously inaccessible. Through this method, a curiosity within the player can be maintained about what might be made available in future play, keeping them motivated to keep on playing.

New items themselves as rewards are yet another popular method, rewarding the player with in-game material depending on the effort and risk that was taken. These items can be more or less valuable or useful, motivating the player to keep seeking out this reward to get a potentially even more epic item.
Resource rewards are similar to item rewards but differ in one key aspect, as they are only a temporary support in the game or application. This can be a one time use health potion to replenish the users hearts or a power up that temporarily improves the abilities of the player character, or grants more experience points for a certain duration. The mechanism of motivation is the same.

The most immediate way to provide a reward is through feedback messages. After the user achieved something inside the application, those messages are directly displayed on the screen, giving an immediate sense of reward and achievement. This can happen through colorful pop-ups that congratulate the player on their success, or in games even through positive reaction of a NPC (Non-Player-Character). Despite the praise being generated by a computer and not coming from an actual human, this still affects human emotions and behaviours, working further to drive the user to complete their tasks.

Rewards like the ones named above are easy to implement and are specifically triggering extrinsic motivation, but there are some ways to trigger intrinsic motivation within the players as well \cite{mmo}.

Transparency around rewards like experience points, unlockable skills and items, and the ability to showcase them to other players creates comparison and therefore competition. This can trigger an inner sense of pride and ambition within the individual players, making them more likely to put more effort into the task at hand.
The reward system has to be viewed as fair and equal to all players. This can be facilitated through the aforementioned transparency, as well as through keeping track of the players advancements in the application, suggesting that there are no hidden ways or shortcuts to recieve the rewards and reach the goal. The keeping track of the successes and advancements of the player itself can equally keep the player motivated, reminding them of past achievements and urging them to forge ahead.
Chance has always been a factor in games that keep the players curious and engaged. If a reward does not have a 100\% chance to be given, players have the tendency to keep trying until the wanted reward is recieved and the thrill of the unknown is keeping them engaged.
In the same spirit as chance, scarcity in rewards triggers the feeling of greed and need for the unlikely reward within the player. Nothing is gained for free,  and the player has to take sizeable risks or efforts to aquire the rare reward. Like in the material world, items that are associated with rarity, like gold, are desirable and people will go to considerable lengths to acquire it.
Giving constraints like these make it more likely for the player to feel a sense of intrinsic motivation, prompting them to keep on playing and engaging with the available material \cite{mmo}.

Generally, there are some base guidelines around implementing reward systems in games and especially gamified applications to motivate users to start and keep playing. The most important aspect to keep in mind is that rewards should be accessible during shorter play sessions. Giving the user feelings of success early on is crucial to promote replayability and make the user more likely to keep logging back in as it lowers the hurdle to start playing \cite{mmo}. If rewards are held off too long, the player might get frustrated and feel obligated to keep playing for a longer time to achieve what they want, which can lead to them abandoning the game due to feelings of stress and pressure.
Competition is yet another point that needs to be very carefully considered, along with the goals the application offers as well as the emotions that it is meant to elicit. This is such a difficult aspect due to the vast variety of natures amongst players, who all value those aspects differently and might have differing reactions to the different uses of these mechanics. Overall, players seem to be more interested in rewards that lead to gaining more power or skills in game than in pure aesthetic items meant as a trophy. As such, rewards that are useful for the advancement in the process are more effective to motivate the users, like power ups, unlocking levels and similar rewards \cite{fail}.
A scoring system overall is an easy way to introduce a transparent system to keep track of ones own progress and successes, allowing the player more ownership over their process as well as keeping them motivated. This can be as complex or as simple as desired, ranging from 'success or no success' to more complicated point systems where a score is obtained through different actions or in different gameplay areas, creating a much more detailed image of the players progress \cite{lifelong}.
As aforementioned, humans give objects different values based on their scarcity. Therefore, a reward for an 'excellent' success or display of behaviour should be rare enough to be valuable without it accidentally triggering frustration and demotivation on the players side through it either being too complicated to achieve or simply be too scarce \cite{lifelong}.

Especially when it comes to reward systems and motivators in gamification specialized for skill acquisition, it's important to remember that the learners themselves usually are their own main motivator with quite some initial intrinsic motivation already present. Through careful choosing of what motivators and rewards to use, this can be amplified to create an engaging learning experience, however this initial motivation can decrease quickly if the player discovers too many disadvantages or sources of frustration within the application, causing them to get bored or to feel overwhelmed \cite{lifelong}. Instead of bombarding the users with many rewards in the hopes to keep them satisfied, it is key to use the available methods smartly and in the necessary instances and instead of overwhelmedness trigger healthy ambition and a sense of pride in the users individual achievement.
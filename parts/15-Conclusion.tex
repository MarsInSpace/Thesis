Having looked at, and compiled much data all about gamification and other gaming techniques, as well as aspects that are crucial to effective learning, there are some Do's and Don'ts for the design of an effective gamified application for skill and knowledge acquisition that have crystalized. As with the development, it is key to be very clear and aware of the purpose, intended content and target group of the application that is being developed, but generally these guidelines apply. Finally, a couple of questions can be asked that can serve as a useful guideline during the development process. For a compiled overview, reference Table \ref{table:2}.

\section*{Do's} \indent
\textbf{Relevant \& Accessible Rewards:}
Well thought through rewards that match the users needs and help them progress. This refers to external rewards within the application that are useful to the user in the near future, such as a power-up for example. Exceptional performances should earn a bonus reward. These rewards should equally be accessible during short play sessions. \\ \indent
\textit{Why:} Through using relevant rewards that are tailored to the target group, and their needs and psychology, extrinsic motivators are firstly more effective, and secondly might help trigger intrinsic motivation within the user. Similarly, offering more exceptional rewards for a good performance motivates the user to strive for improvement, therefore making the learnings more effective. Allowing the user to feel rewarded and successful even during short play sessions hightens engagement and the likelyhood of the user returning more frequently.

\textbf{Storytelling \& Immersion:}
Implementing some sort of narrative within the application, either through the actual telling of a story or through situating the content of the application within a fictuous context. Additionally, to enhance immersion, it is possible to provide some sort of roleplay or fictious interactions for the user within the application or even allow for the creation of an avatar or similar virtual incarnation for the user to identify with. \\ \indent
\textit{Why:} Through storytelling, the application may engage the users on a deeper level, making them interested in finding out what is next or what else can be explored. This game design element can even trigger deeper intrinsic motivation within the user which is beneficial to effective learning. The use of an avatar can be a good choice to further make the application feel like play, permitting the user to immerse themselves further and therefore have a personal connection with the content, equally fostering engagement and an intrinsic drive.

\textbf{Well designed Graphics \& Audio:}
Visually pleasing assets and a cohesive, overarching graphic design for the application with a friendly, direct, and easy to understand interface. Simple but effective sound effects for feedback. \\ \indent
\textit{Why:} A visually appealing application is more likely to be used and to remain continuously used. Good visual design as well as a simple design for the overarching game interface increases the satisfaction of interacting with the app and avoids frustration with the handling of the application itself. The concious use of satisfying sound effects increase the feeling of success when recieved and further make the application usage fun and stimulating.

\textbf{Feedback \& Affirmation:}
Similar to rewards, providing immediate feedback for the users performance and actions, keeping the application responsive to interactions and help users keep track of their progress and improvements. \\ \indent
\textit{Why:} Giving the user immediate feedback after their actions will help motivate and guide them through the content of the application, and implementing for example a scoring system to track their achievements and progress helps fostering confidence in their improvements and keeping them engaged. Additionally, having audiovisual cues when a task is completed or a goal achieved is a satisfying reward for the user, increasing the fun-factor for the task and therefore the application as a whole.

\textbf{Interaction \& Levels:} 
Design the application with many levels, either for tasks, stages of knowledge, skill levels, etc. Additionally, achieve a high density of interaction through interactive tasks and step by step completion, making the user a participant and not an audience. Gradually allow the user to gain access to the next steps, for example unlocking new levels or introducing new mechanics. \\ \indent
\textit{Why:} Completing a level is a big motivator for users. Separating larger tasks and topics into smaller, more easily digestible parts and creating levels around it will keep the user engaged, entertained and motivated through the feeling of success they get by finishing a level. Making those levels as interactive as possible not only keeps the users attention but also facilitates learning through direct contact with the topic instead of just absorbing information. The gradual unlocking of content will keep the user curious as to what might come next and will make it more likely for them to keep on using the application.

\textbf{Well defined Goals:}
Provide concrete, feasible short, medium and long-term goals. Be very clear about what is expected from the user and how to achieve the set goal. Especially in a learning context, set clear, realistic learning objectives. If possible, make the goals in the application align with the personal goals of the user. \\ \indent
\textit{Why:} Clear direction in what to do helps the user feel secure in their actions and gives them a clear prospective on the process and goal they are supposed to achieve. Implementing milestones as short and mid-term goals helps measure progress and improvements, motivating the user to keep going while giving them the feeling of success they need to feel like they achieved something relevant for their own personal development. Especially if the application is tailored to a specific need or want of the user, for example, to learn a language the goals can be aligned with their own and therefore provide a strong internal drive.
Clear and transparent goals also prevent frustration that can arise from dealing with a task or topic that is too complex and seemingly without an end to it.

\textbf{Clear Rules and Parameters:}
Provide a very transparent set of rules and parameters as well as a clear and efficient onboarding to instruct the user on how to use and progress within the application properly. \\ \indent
\textit{Why:} Not knowing how exactly an application works and having the rewards and feedback feel arbitrary very quickly triggers frustration and will demotivate the user. Having clear instructions on what to do will help the user achieve the goals better and faster while also helping them comprehend the content more easily.

\textbf{Challenge \& Increasing Difficulty:}
Create sufficiently challenging levels or tasks that over the course of the progression within the app become more difficult, staying synchronised with the increase in skill and knowledge of the user. Depending on the content and if it is compatible with different degrees of difficulty, having adjustable difficulty levels can be beneficial for the user. Additionally, make sure the user can repeat the tasks and levels at their own leisure. \\ \indent
\textit{Why:} A consistent sense of challenge that is still reasonable and manageable can encourage the ambitions of the users, keeping them motivated to compete further with themselves by solving increasingly complex problems and giving the satisfaction of completing them. Furthermore, increasing difficulty is needed to have an efficient learning process, gradually introducing new knowledge and providing the tools for the user to acquire said knowledge and excel at it. Making repetition available will give the user the opportunity to revise knowledge and deepen their understanding of the topic.

\textbf{Novelty \& Variety:}
Provide new content regularly, offer different types of playful learning and not just memorization excercises. Switch up the type of interactions regularly. \\ \indent
\textit{Why:} If the content and the interactions stay the same, the user will loose interest due to boredom. This also does not promote learning as the content stagnates. Providing different methods of acquiring the knowledge and switching them up will keep the user engaged and curious, as well as foster their adaptability and problem-solving skills.

\textbf{Choice \& Exploration:}
Promote decision making by offering a flexible path selection, allowing exploration and giving the user the power to set their own pace. Allow the user a certain freedom of deciding what task, problem, or level they want to tackle next. \\ \indent
\textit{Why:} Giving the user a certain amount of autonomy within the process encourages them to take control of and feel responsible for their progress without feeling pushed or restricted by the application. Being able to choose the next step and the content they want to engage with will encourage motivation and circumvent the reluctance that the user might feel if the path the application offers is too rigid and might make uninteresting content unavoidable and a barrier to move forward. Allowing users to set their own pace in learning equally prevents them from feeling overwhelmed and encourages them to keep trying and eventually progressing.

\textbf{\emph{Situational:} Optional Social Interaction:}
Have opt-in elements for social play within the application like leaderboards, a friendlist, the forming of parties, etc. \\ \indent
\textit{Why:} Some people strive with social factors within the applications they interact with. This is highly individual and only recommended if it is very clear which target group the application is developed for and how their stance is on this type of interaction. Make sure to research thoroughly before implementing this.

\section*{Don'ts} \indent
\textbf{Changing Rules:}
The sudden change of rules and parameters that have been a certain way for the entire process up until that moment, but suddenly are different for no apparent reason. \\ \indent
\textit{Why:} If suddenly the ruleset changes, it is disorienting to the player. They might not understand why and how exactly the existing rules were discarded or adjusted, and might have a difficult time adapting to the new circumstances. Frustration might arise as a concequence and hamper motivation and engagement. \\ \indent
\textit{What to do instead:} Design and communicate a clear ruleset at the beginning that will not change throughout the process.

\textbf{Too overwhelming or not stimulating enough:}
The tasks or levels are either too easy or too difficult. Too much new content is introduced at once. The onboarding is too long and complex. \\ \indent
\textit{Why not:} If the tasks or levels are too easy, the user will quickly feel understimulated and feel a lack of challenge, which will lower their ambition, get bored and decrease their motivation to engage with the application. On the other hand, if the tasks are too difficult or too overwhelming by, for example, introducing too much new information at the same time, the user might feel helpless and lost within the task, or get frustrated by not achieving the necessary goals to progress. Additionally, the first hurdle within the application is the onboarding process. If it is too difficult to understand and takes too long to explain, the user will loose interest right away and might cease to use the application before even having properly explored it. \\ \indent
\textit{What to do instead:} Have a good balance of well designed tasks, building on previously gained knowledge to permit the user to solve it without too much difficulty while still introducing new content bit by bit, providing a comfortable level of challenge. Keep the onboarding concise and clear.

\textbf{Restrictive path \& pace:}
Rigid structure of how a task is to be completed without much room for exploration or alternative methods. Needing too much repetition of one task or level before allowing the user to move on to the next or not allowing any repetition at all. \\ \indent
\textit{Why not:} Designing the structure of the application in a very rigid and restrictive way might make the user feel caged and forced to do things a certain way without giving them the opportunity to take a little control over their process. This mostly just leads to frustration and reluctance to continue as the dictated path and pace might not be compatible with the learners individual needs. Through too much repetition the user will get bored, while allowing no repetition at all might not give the user enough time to properly integrate the subject. \\ \indent
\textit{What to do instead:} Allow a selection of a few levels or tasks that can be tackled next instead of forcing a specific one. Let the user decide whether they want or need to repeat the task or level before moving on to the next one, and allow them to return to the level and re-play it at any moment.

\textbf{No Feedback on Progress:}
Lack of progress bars, levels, any measurement of performance and progress. \\ \indent
\textit{Why not:} If no means are given to the user to keep track of their own achievements and progress, there is no possibility to notice improvements or advancements, taking away a crucial motivator. Additionally, a lack of indicators about the users score or ability doesn't offer a way to allow them to compete with themselves, therefore not fostering the ambition within the user to perform even better in the next task, or by repeating the same level. \\ \indent
\textit{What to do instead:} Implement a very transparent system that allows the user to keep track of their skills and knowledge improvements.

\textbf{Bad Reward Systems:}
No rewards are given at all. Too many rewards or rewards that are unrelated to the application or to the tasks the user needs to complete. Too much focus on extrinsic motivators or competitive rewards. Only using badges, points and leaderboards as a reward system. \\ \indent
\textit{Why not:} Poor quality rewards do not achieve the motivational boost within the user that is required to create an engaging experience. Rewards that are unrelated to what the user is currently doing and have no actual use within the application except cosmetic are proven to not foster motivation and might even frustrate the user. Badges, Points and Leaderboards, while perhaps initially motivating, do not provide long term motivation and therefore users might loose interest very fast. Similarly, if only extrinsic rewards are given without a system to help promote intrinsic motivation in the user as well, users are less likely to stay driven and engaged. \\ \indent
\textit{What to do instead:} Give rewards that matter and are useful to the user, for example unlocking further levels or power ups that give them a temporary boost or advantage within the application. Try to find ways to foster intrinsic motivation through well designed external rewards and other feedback.

\textbf{Punishment for Failures:}
Loss of a life, health, points or similar when a level or task is failed. General punishment for not performing well. Loss of advantages through not maintaining a daily log-in streak. \\ \indent
\textit{Why not:} Failure in itself is already a stressor for users, if the application amplifies that feeling of failure through negative consequences, users might feel demotivated, frustrated, disappointed in themselves, and might put less effort into working with the application or stop using it completely. Showing that failure is not something bad and can be seen as an opportunity to try again and get better is key in providing an effective and healthy learning process and therefore this kind of negative feedback should be avoided. \\ \indent
\textit{What to do instead:} Give incentives to try again. Give rewards for repeating a failed level. Give rewards every time the user logs in and do not take them away when they fail to do so.

\textbf{Obligatory Social Interactions \& Competition:}
Competitive elements like leaderboards or direct battles, as well as other social elements like mandatory participation in parties or guilds or having to play with friends as fixed component of the gamified application. \\ \indent
\textit{Why not:} While some forms of competition and social play can be quite motivating for some people, most will perceive it as stressful and demotivating, especially if the user might not have as good a high score or performance as others, showing them their lack of skill as a lower spot on the leaderboard or similar. Social play that comes in the form of daily interactions that are either mandatory or would give a massive disadvantage if missed can cause the user to start perceiving the usage of the application as a chore or as work since they are required to log on every day or suffer penalties. Neither is good for fostering motivation or engagement. \\ \indent
\textit{What to do instead:} If a social component has to be present, keep it optional and opt-in. Let the users decide if they want to compete or otherwise interact with strangers or with friends.

\textbf{Time Limits:}
Imposing time limits on the completion of the levels or tasks or similar. \\ \indent
\textit{Why not:} Most people percieve a time limit as stressful and it will impact their performance negatively. Although this method is effective for some people, it is safer to leave out this specific game mechanic. \\ \indent
\textit{What to do instead:} Allow the user to set their own pace. Set limits to how many actions can be taken to make a problem or task more challenging, if any.

\section*{Questions as Guidelines for Development}
\begin{itemize}
    \item What are the specific goals that the gamified tool is supposed to achieve?
    \item What behavioural goal is the focus? Increased motivation, engagement, enjoyment or improved achievement?
    \item Are there experts on the team for the content of the application (eg. Instructional Designers)?
    \item What is the target audience and how does that impact the choice of game elements?
    \begin{itemize}
        \item Learning styles, characteristics \& preferences
    \end{itemize}
    \item How can the knowledge about the future users be leveraged to cultivate their intrinsic motivation?
    \item Which extrinsic rewards are useful and effective for the purpose of this application?
    \begin{itemize}
        \item Are the quality of rewards favoured over the quantity of them?
    \end{itemize}
    \item What is a fitting, satisfying visual style for the application?
\end{itemize}

\renewcommand{\arraystretch}{1.5}
\begin{table}[h]
\centering
\small
\begin{tabular}{ |p{7cm}|p{7cm}|  }
\hline
\textbf{DO'S} & \textbf{DON'TS}\\
\hline
\begin{itemize}
    \item Relevant \& accessible Rewards
    \item Storytelling \& Immersion
    \item Well designed Graphics \& Audio
    \item Feedback \& Affirmation
    \item Interaction \& Levels
    \item Well defined Goals
    \item Clear rules and parameters
    \item Challenge \& Increasing Difficulty
    \item Novelty \& Variety
    \item Choice \& Exploration
    \item \textit{Situational:} Optional Social Play
\end{itemize} &
\begin{itemize}
    \item Changing Rules
    \item Too overwhelming or not stimulating enough
    \item Restrictive path \& pace
    \item No feedback on progress
    \item Bad Reward System
    \item Punishment for Failures
    \item Obligatory Social Interactions \& Competition
    \item Time Limits
\end{itemize} \\
\hline
\end{tabular}
\caption{Do's and Don'ts for implementing Game Elements in a Gamified Learning App}
\label{table:2}
\end{table}
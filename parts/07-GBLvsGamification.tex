There are many different types of games and game aspects that are combined with or used for education. Gamification and Game-Based Learning are often used as Synonyms for simplification purposes although they describe different strategies of using gaming for more than enjoyment purposes. Table \ref{table:1} is specifically comparing gamification and Game-Based Learning, but there are more keywords one may stumble across while researching in this field:

\textbf{Game:}
As a fundament to every other more specific type of games in skill or knowledge aquisition lies the question, what exactly is a game in the first place.
According to Merriam Webster \cite{fail}: 
\begin{quote} 
    A game is defined as: a. A form of competitive activity or sport played according to rules;
    b. An activity that one engages in for amusement; 
    c. (adj) eager or willing to do something new or challenging. 
\end{quote}
\indent
\textbf{Gamification:}
The definition of gamification is widely accepted as \cite{higher}: 
\begin{quote}
The use of game elements to create an active learning environment by engaging the learner in the process of knowledge acquisition.
\end{quote}
or as \cite{edu}:
\begin{quote}
The use of Game Mechanics, Aesthetics and Thinking to engage, motivate action, promote learning and solve problems.
\end{quote}
or just as \cite{higher} \cite{edu}:
\begin{quote}
The use of game design elements in non-game contexts.
\end{quote}
Gamification is inspired by video games and their ability to keep players engaged for long periods of time without losing interest \cite{equilibrium}. Through thorough analysis of what exactly makes games fun, developers isolate these aspects and apply them outside of gaming - for example in marketing, learning, or even in doing mundane tasks - to create a feeling of fun or even addiction to influence the users behaviour \cite{edu} \cite{fail}.
Although Gamification is also being used outside of educational or learning contexts \cite{compare}, game mechanics and gameplay elements are often applied to existing learning content.
Examples for such elements include: Achievement Badges, Points and Leaderboards, Progress Bars, Levels and Quests \cite{compare}.

\textbf{Game-Based Learning (GBL):}
Game-Based Learning or GBL for short uses games as part of the learning process \cite{compare}. Instead of trying to simulate an entire learning journey or enhance the experience thereof, Game-Based Learning isolates a single learning objective and turns it into a game.
The process for this can either be building new games from scratch to finetune it to the exact needs of this learning objective, or using existing games for educational purposes, a very common example for teaching economics and history is Sid Meyers Civilizations, see Figure \ref{fig:7}, that had been developed for non-educational purposes and was later refashioned into a learning game \cite{fail}. 

\textbf{Educational Games:}
Educational games are all games that are being designed and used with the intention to use them to teach and to learn \cite{compare}.
It tries to combine elements of fun and educational concepts with the goal of increasing student motivation and engagement \cite{compare}.
An educational game is built specifically with education in mind \cite{fail}. 

\textbf{Serious Games:}
Serious games are games designed for the specific purpose of training a group or individual for a specific task or job, not just for fun or entertainment, often used in medicine, military or for dangerous manual labor \cite{edu}.
A serious game possesses all game elements, it looks like a game, but it has a predetermined objective \cite{edu}. 

\textbf{Simulations:}
Simulations are similar to serious games but instead of just having a specific training purpose, they are made to simulate real-world situations and have the goal of offering the user a training environment that resembles real life \cite{edu}. 

\textbf{Game-inspired Design:}
The core of game-inspired design is the use of ideas and ways of thinking that are inherent to games and game development. It is not the use of game elements that defines this method, but rather the use of playful design in the final product \cite{edu}. 

\renewcommand{\arraystretch}{1.5}
\begin{table}[h]
\centering
\tiny
\begin{tabular}{ |p{3cm}||p{4.5cm}|p{4.5cm}|  }
\hline
\textbf{Comparison Points}& \textbf{GBL} & \textbf{Gamification}\\
\hline
Definition & - The use of games to enhance a learning experience & - The use of game design elements in non-game contexts \\
& - The use of a game to teach about a topic & - The use of game mechanics, aesthetics, and game thinking to engage, motivate action, promote learning and solve problems \\
& & - The use of game elements to create an active learning environment by engaging the learner in the process of knowledge acquisition \\
\hline
Purpose & - Teaching about one subject through a game & - Engaging and motivating the User to fulfill a task \\
& - Motivate students through gaming & - Turn the learning process into a fun experience \\
\hline
Benefits & - Skill-building through a cohesive game experience & - Very versatile, can be used in many different contexts \\
& - Keeps users motivated through ongoing action & - Better learning experience \\
& - Other benefits of general gaming like hand-eye coordination \& strenghtening decision making & - Motivates through Reward Systems \& Playfulness \\
\hline
Interaction with the Learning Process & - One learning aspect is isolated and taught & - The entire (learning) process is gamified \\
& - The learning goal is usually morphed to fit the game & \\
\hline
Use of game elements & - An entire game is used & - Game elements are selected and used to enhance the existing (learning) material / task \\
& & - Reward systems \\
\hline
Examples & - SimCity & - Duolingo \\
& - Sid Meyer's Civilizations & - Kahoot! \\
\hline
\end{tabular}
\caption{Comparison of GBL \& Gamification}
\label{table:1}
\end{table}
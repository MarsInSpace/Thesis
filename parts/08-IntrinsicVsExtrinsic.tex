\textbf{Intrinsic Motivation:}
The definition of intrinsic motivation has been described in many different ways but mainly follow the same core principle of it coming from the person itself, it comes from enjoying the activity or by having a large interest in it with little to no external influence \cite{corners} \cite{equilibrium}.
It is the pride one takes in doing the task, or the simple joy of the action \cite{domestic}, like an internal 'want to' drive \cite{equilibrium}.
Intrinsic motivation provides the reason for taking part in the game and is different than from what the gameplay demands \cite{mmo}.
Game designers and users alike favour intrinsic motivation due to it's fast pace and powerful feeling of reward, usually keeping players engaged longer \cite{corners}.
An example for an intrinsically motivated task within a game is the solving of a riddle to access a new area. The player isn't rewarded directly by the game but an internal feeling of a job well done motivates the player to forge on \cite{aspects}.

\textbf{Extrinsic Motivation:}
Extrinsic motivation can be defined as external reward or reinforcement \cite{domestic} or as a motivation to win in order to get a desirable reward \cite{corners}.
Examples for extrinsic motivation can be that a reward is given when the correct solution was provided \cite{mmo}, or the gaining of points to appear on a worldwide leaderboard \cite{aspects}. In many games, the slaying of monsters is rewarded with experience points or items, this also counts as extrinsic motivation \cite{corners}. Outside of the video game context, a student studying solely to earn good grades or to avoid failure, can also count as an instance of extrinsic motivation \cite{equilibrium}.
Despite it's short term effectiveness, relying solely on extrinsic motivation can cause some negative effects, as it tends to be very time consuming and a very unpersonal type of reward. Additionally, the rewarding feeling doesn't tend to last as long, which might dampen long-term motivation. Lastly, players set high expectations for themselves when extrinsic rewards are involved, leading to more negative thoughts towards loosing than normally \cite{corners}.

\textbf{Comparison:}
Intrinsic and extrinsic motivation often go hand in hand but differ in the fundamental drive that lays underneath. One could say that intrinsic motivation materialises in an "I want to..." drive, while extrinsic motivation materialises as an "I have to..." drive \cite{equilibrium}.
Often, by providing extrinsic rewards, intrinsic motivation might trigger as a concequence \cite{mmo}.

\textbf{Context in Games:}
Playing a game is often intrinsically motivated (the player wants to play the game) and the results are extrinsic (rewards inside the game give the player a feeling of success) \cite{domestic}.
Intrinsic motivation is widely seen as the most important educational aspect of digital games \cite{domestic} and is often used in more recent popular game genres to generate more enjoyment for the player on a short term basis \cite{corners}, although most games rather provide unrelated, extrinsic rewards due to it's easier implementation \cite{fail}.
Researchers agree that relying solely on extrinsic motivators does not create effective gamification \cite{equilibrium}.
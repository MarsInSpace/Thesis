No matter the age, culture, social background or ethnicity, people have always loved to play games \cite{framework}.
Saying that games are an attractive medium throughout the population regardless of gender, age, and other social factors is an understatement \cite{mmo}.
The Entertainment Software Association conducted a survey which found that at least 67\% of American head of households play computer or video games on a regular basis and other studies have been done, showing that there is considerable interest and curiosity amongst the learner population to use games and gamified tools for skill aqcuisition purposes due to their known ability to engage players and keep them motivated \cite{engage} \cite{framework}.
Despite their popularity, formal education systems keep a sceptical view towards games in education, doubting their efficiency and painting them in a generally unserious light \cite{lifelong}.

Contrary to how it might be percieved, digital games are not new at all to the field of education. Educational games and Edutainment\footnote{Edutainment according to the Cambridge Dictionary: 'the process of entertaining people at the same time as you are teaching them something, and the products, such as television programmes or software, that do this' \cite{cambridge}} represent 7-9\% of the video game market \cite{engage}.
Current modern pedagological trends in education are a good opportunity to introduce new digital approaches to traditional and long-established teaching methods in order to enhance active learning and encourage new approaches. The ever-increasing use of Information and Communication Technology in learning context reinforces why the time to double down on introducing these approaches is now \cite{edu}.
Additionally to its popularity on smaller scale levels like in schools and in homes, Game-Based Learning has been taken more seriously on a political level as well, the European Union actively supporting and initiating multiple projects on an EU-wide scale like the Elektra Project (2009) - an adventure game designed to keep up with commercial games with a focus on curriculum-related educational purposes by incorporating a sound psychological and pedagogical framework \cite{elektra} - and 80 Days (2009) - a video game inspired by Jules Verne's novel "Around the world in eighty days" with the goal of creating a game that integrates intelligent and pedagogically proven personalisation and provides interactive and individual storytelling \cite{eightydays} \cite{lifelong}.

Many academic papers seem to agree that despite decades of research on games used for educational purposes, there is a considerable lack of appropriate and interesting media and content that engages learners and improve the learning process or add to it. Furthermore, besides the lack of actual good content, there is a notable shortage of guidelines or frameworks for effective educational game design models, causing instructional designers and game designers alike to develop their applications with a trial and error approach \cite{aspects} \cite{framework} \cite{online} \cite{model} \cite{equilibrium}.
One paper claims additionally that another issue that this sector of research has, is the inconsistency in the outcomes of empirical research regarding Gamification in learning environments, leading to inconclusive results about the efficiency of their usage \cite{equilibrium}, however others claim that there are plenty of studies by various researchers that indicate the opposite, showing that many strategies, tactics and methods used in popular gaming can provide valuable lessons and tools for the design of instructional media \cite{engage}.
Despite researching the overall effectiveness of games and gamification in instructional contexts, research has often ignored the actual methods and tactics that popular game design uses to engage their players and how these might be translated to educational game design \cite{engage}.
Past research also mostly focussed on specific game genres and their design, making their finds and suggestions harder to use and implement when the target game genre of the new product differs to the one in the study \cite{model}.

In educational environments as well as in the general environment of skill acquisition, there is a growing demand for the learning materials to be more interactive as well as offer individualisation and customisation of such materials to cater to todays learners needs \cite{aspects}.
Game-Based Learning is already being widely used in childrens education all over the world \cite{aspects} and an increasing number of countries - specifically in northern europe like Norway or Sweden - are implementing a higher focus on digital skills in their education system, routinely using either full games or gamified tools like Quiz-games in their curriculum and teaching methods \cite{domestic}.

Overall, digital games as a means of learning have gained more importance due to changes in modern infrastructure and the society as a whole \cite{lifelong}.
The lives of people of newer generations are significantly more entwined with the digital world, be it at home, in schools or in the workplace, making them so-called "digital natives", adept at navigating the web and other technologies with ease. This, as well as the huge popularity of video and computer games throughout all generations and populations, are reasons as to why new ways of learning have to be discovered and developed to stay up-to-date with the needs, trends and capabilities of individuals in modern society \cite{aspects} \cite{domestic} \cite{online} \cite{framework}.
Due to those ever-changing consumer demands, game designers are continuously looking for new ways and strategies to engage their players \cite{mmo}.

Though there are overarching similarities between commercial games designed with the sole purpose of entertaining the player and games specifically designed for educational purposes, studies make an important distinction between them in their research \cite{domestic}.
Despite both being on the rise in their respective fields and industries, as previously mentioned, e-learning and digital media in the educational field have been under heavy scrutiny by academics and educators alike due to a number of current limitations but it seems that the academic debate about the educational value and potential of such content is decreasing and instead shifting focus towards questions found in the development of such media, namely the cost of development, the complexity and challenges in implementing said media into the curriculum, or how to assure the quality of the learning process \cite{online}.
Since this topic has been gaining notably more momentum in popular debates, a growing number of tools are being created to facilitate the implementation of games and gamification into current learning contexts \cite{online}.

Game-Based Learning and Gamification are models that commonly get seen as a 'cure-all' for the problems of traditional education due to the popularity of gaming in society \cite{traditional}. Although this view is very much blown out of proportion, one can't ignore the aspects that gamified learning can bring to the table. Studies show that educational games provide immersion, motivation and fun, as well as provide content that the user can relate to and that is rooted in reality, teaching 21st century skills all while serving its original purpose \cite{framework}.
There are many studies that show pleasure and fun to be inherent aspects of games, and systems specifically for designing games for a learning environment context use those as core motivators to improve learner's engagement \cite{compare}.

Despite instructional designers and game designers best efforts, educational games often lack the engagement and fidelity that they want to copy from commercial games. Vice versa, commercial games generally fall short on intellectual content that truly teaches subject matters in a constructive and pedagogically sound way.
To create meaningful and effective tools for skill acquisition, it is crucial to find ways to combine the two, pick and choosing which elements of which can enhance the overall performance and utility of the tool to develop powerful and motivating gamification \cite{traditional}.
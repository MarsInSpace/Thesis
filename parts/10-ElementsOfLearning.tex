To provide a successful learning experience in and out of gamification, there are several factors that need to be taken into consideration. There are several approaches and sets of principles and goals that should be focused on to create a sustainable and productive learning environment, allowing for an effective acquisition of skills.

Unrelated to which medium is being used to teach the required skill, there are a few aspects that are key to teaching and should be integrated in the learning process to facilitate a fruitful learning experience:

\textbf{Acquisition of information and practice/repetition:}
First and foremost, providing the content for learning and practice is the key of any skill acquisition process. Study materials and a structure must be given to start any learning process in any environment to allow for the acquisition of information. Furthermore, the space for practice and repetition of the learned content helps integrate the gained knowledge in a more sustainable and permanent manner \cite{lifelong}.

\textbf{Ownership in learning process:}
Having the student actively partake in the shaping of the learning process encourages them to take ownership over their own process, allowing for more investment and motivation to learn the skill at hand \cite{aspects}.

\textbf{Adapted pace that allows reflection and thorough study:}
Allowing the student to move at their own pace increases the efficiency of the learning process and the process of skill acquisition as a whole. Having an adapted pace to the individuals needs allows for and encourages reflection and the possibility to study the topic thoroughly instead of being rushed through the lessons with minimal understanding or getting bored by too slow a pace \cite{online}.

\textbf{Social experience, collaboration \& relation to real life/relevant contexts:}
Authenticity is a major factor in successfully teaching a new skill. Embedding the learning within a relevant context and with relations to real life situations or having a simulation that alludes to it allows the student to integrate the new knowledge in a useful way that makes sense to them instead of learning said skill detached from reality \cite{aspects}.
Encouraging collaboration with other students within relevant learning processes can enhance the efficiency of learning. Through discussion with others, the topic gets integrated more thoroughly and additionally, it grants the student social experience and communication skills \cite{lifelong}.

\textbf{Training of problem-solving:}
Encouraging and allowing the individual student to tackle the posed problem or task and using their problem-solving skills is crucial to develop those skills further and can be used intersectionally and generally in life, not just the current topic of study. Getting better at problem-solving will further enhance the learning process and allow the student to come to some conclusions by themselves which provides them with a feeling of success and additional intrinsic motivation to continue studying \cite{lifelong} \cite{online}.

\textbf{Encourage multiple perspectives/intersectionality \& awareness of self and environment:}
For the better integration of the acquired knowledge or skill, the referencing of intersectional instances and encouragment to take on multiple perspectives on a topic will help the student think outside the box and create a wider network of knowledge for themselves as they are able to use the new information in a broader area, or even in contexts that were seemingly unrelated. Similarly, the knowledge or skills from other areas of expertise might support the student in learning this new particular skill. This generally enhances the students awareness of themselves as well of their environment, learning to look deeper than what is obvious on the surface and creating processes and connections that might have been overlooked otherwise \cite{aspects}.

\textbf{Progression \& facilitating success in the learning activity:}
In any learning environment it is important to have a visible sense of progression inside the process to keep the learners engaged and to facilitate success in the learning activity or respectively facilitate ongoing opportunities for a feeling of success within it to consistently show the students progress and motivate them further \cite{fail} \cite{equilibrium}.

\textbf{Assessment of progress:}
In traditional education especially but equally in any other learning medium, the assessment of progress is very important to further develop the learning process, to find out where more repetition is needed and which skills or topics have already been mastered. This allows for a more individual tailoring of the instructions or the following tasks to progress in the learning process and therefore a more effective path to acquiring the desired skill \cite{online}.

\textbf{Freedom to fail:}
Lastly, communicating to and showing that the student has the freedom to fail and the unhurried and non-judgemental space to try again in the case of failure is key to creating a non-hostile learning environment which encourages resilience and the willingness to keep trying even in the face of more complicated problems. This also encourages the exploration of different types of approaches, facilitating intersectionality, the taking of multiple perspectives as well as training the students problem-solving ability further \cite{fail}.

Games generally already embody a lot of these attributes, allowing for an engaging experience that can be translated to other teaching approaches. Some mechanics intuitively integrate processes and systems that appeal to the user, creating a motivating environment that fosters skill development due to the nature of those games as a whole, even in commercial contexts.

\textbf{Interaction:}
The core feature at any game and gamified tool or application in general is interactivity. For anything to happen or to move forward, the user is actively required to take action, take decisions and responsibility for setting their own pace \cite{framework}. This already tackles some main points of the learning principles listed above: Through the high adaptability of the process as a whole, the user takes ownership over the pace and part of the content, being able to spend more time on specific tasks that are of interest to them, making it more interesting and engaging.

\textbf{Structure \& Progression:}
Generally, games offer an inherent structure and clear path of progression, through their storylines, by completing puzzles or challenges, and are integrating those into a coherent context that makes the actions the player takes feel meaningful and keeps the player motivated to grow and improve further \cite{framework} \cite{fail}. Additionally, the progress tends to happen at the players own pace, allowing them to explore, come up with creative solutions that might help with solving future problems as well, and not feel overwhelmed by an externally enforced pace \cite{framework}.

\textbf{Challenge:}
Compelling tasks and challenging problems that provoke the players ambition and require at times complex problem-solving skills are a key aspect of many games today. Games have mastered the skill of creating those challenging problems while still keeping them doable, creating a pleasantly frustrating experience that highly motivates the players \cite{framework}.

\textbf{System thinking, changing perspectives \& intersectionality:}
Games generally require players to look at the bigger picture instead of isolated events or causes. This encourages them to look at a problem from different perspectives and try different approaches, potentially connecting the solutions to previous problems to the one at hand. Thorough exploration is at times not just encouraged but required instead of rushing through the game, allowing the player to properly wrap their head around the problem and potentially adapt their approach or even their goal \cite{framework}.

\textbf{Freedom to Fail \& room to change your mind:}
It is always possible to try again in games. When one approach to a problem doesn't work, players are encouraged to explore and find alternative ways to get to a solution. Some games are even centered around the idea of integrating failure into their game-loop, at times requiring the player character to die to get a certain outcome. This teaches resilience and that failure is not defeat, but the opportunity to learn something new and to adapt to new ways of doing things \cite{fail} \cite{framework}.

\textbf{Storytelling - hook \& induction into relevant context:}
Many games embed their puzzles and challenges into a broader context, usually facilitated by a storyline that is told one way or the other in the games world. Storytelling is a powerful tool to make the player feel engaged and interested in the task at hand, as well as to make the tasks and problems make sense within a context that doesn't feel detached or meaningless, inducing a more overarching sense of accomplishment than for example unrelated mini-games would \cite{fail} \cite{aspects}.

\textbf{Social thinking \& Team Work:}
Multi-player games often require players to work in teams, pushing them to think beyond their personal needs and goals, instead working together at times with strangers to find solutions and strategies to win the game, fostering social and group-oriented thinking as well as improving communication skills to quickly coordinate their team, and teaches them to think on their feet and improvise in sometimes high stress situations \cite{framework}.

\textbf{Provides clear feelings of success / Reward Systems:}
Games can clearly trigger intrinsic and extrinsic motivation in their players. Through different reward systems the user gets feedback on the completion of their tasks, keeping them entertained and engaged \cite{equilibrium}.

\textbf{Feedback:}
Additionally to the reward systems integrated in the games, there is generally a lot of feedback being given within the gameplay itself, providing the players with subtle direction that gives them a satisfying sense of knowing what they are doing \cite{framework}.

When developing a tool with the goal of effectively gamifying a learning process, it is crucial to keep those aspects and principles in mind. Gamification mechanics differ in a lot of ways from the mechanics in commercial or educational games, since instead of offering a whole, closed experience, the goal is to take an existing process, and add gamified mechanics and elements to it to make it more engaging and motivating. However, a lot of the basic concepts of these mechanics can be adapted for gamification as well.